% Created 2024-09-23 Mon 13:35
% Intended LaTeX compiler: pdflatex
%%% TeX-command-extra-options: "-shell-escape"

\documentclass{iacrtrans}
\usepackage[utf8]{inputenc}
\usepackage[T1]{fontenc}

% -- Default Packages --
\usepackage{graphicx}
\usepackage{longtable}
\usepackage{wrapfig}
\usepackage{rotating}
\usepackage{tcolorbox}
\tcbuselibrary{skins}
\usepackage{booktabs}
% \usepackage{multirow}
\usepackage[normalem]{ulem}
\usepackage{amsmath}
\usepackage{amssymb}
\usepackage{capt-of}
\usepackage{hyperref}
\usepackage[nameinlink]{cleveref}
\crefname{algocf}{alg.}{algs.}
\Crefname{algocf}{Algorithm}{Algorithms}
\usepackage{tikz}
\usetikzlibrary{shapes.geometric, arrows.meta, positioning, trees}
\usepackage{forest}
\forestset{path highlight/.style={blue,edge={blue, line width=2pt},
for ancestors={edge={blue,line width=2pt},
  if level=0{}{blue}}}}

\newtcolorbox{scriptPubKeyBox}{
  enhanced,
  boxsep=1pt,
  arc=0.75ex,
  colback=gray!10,
  colframe=gray!40,
  boxrule=1pt,
  leftrule=80pt,
  top=-3.5mm,
  overlay unbroken and first ={%
    \node[minimum width=1cm,
      anchor=south,
      font=\sffamily\bfseries,
      xshift=40pt,
      yshift=-6.5pt,
    black]
    at (frame.west) {scriptPubKey:};
  }
}

\newtcolorbox{scriptSigBox}{
  enhanced,
  boxsep=1pt,
  arc=0.75ex,
  colback=gray!10,
  colframe=gray!40,
  boxrule=1pt,
  leftrule=80pt,
  top=-3.5mm,
  overlay unbroken and first ={%
    \node[minimum width=1cm,
      anchor=south,
      font=\sffamily\bfseries,
      xshift=40pt,
      yshift=-6.5pt,
    black]
    at (frame.west) {scriptSig:};
  }
}

\newcommand{\mycomment}[1]{}
\newcommand{\elem}[1]{\, \langle #1 \rangle \,}
\newcommand{\opcode}[1]{\, \texttt{#1} \,}
\newcommand{\script}[1]{ $\big\{ #1 \big\}$ }

% -- Algorithms --
\usepackage[
    titlenumbered,
    linesnumbered,
    ruled
]{algorithm2e}
\SetKwInOut{Input}{Input}
\SetKwInOut{Output}{Output}
\SetKwInOut{Return}{Return}

\SetKwComment{Comment}{/* }{ */}

\DeclareMathOperator*{\argmax}{arg\,max}
\DeclareMathOperator*{\argmin}{arg\,min}

\author{Denys Riabtsev, Mykhailo Khotian}
\institute{Distributed Lab}
\title{Optimized ZK verification of ECDSA signature with partially outsourced computations}

\usepackage{biblatex}
\addbibresource{refs.bib}

\begin{document}

\maketitle


\section{Introduction}
The Elliptic Curve Digital Signature Algorithm (ECDSA) is one of the most widely adopted signature schemes, particularly in blockchain systems. 
However, when it comes to zero-knowledge proof systems, verifying ECDSA signatures presents significant computational challenges.

Current implementations of ECDSA signature verification in zero-knowledge circuits face several limitations. 
The primary challenge lies in the computational complexity of the verification process, which requires numerous elliptic curve operations and modular arithmetic calculations. 
These operations translate into a large number of constraints when expressed in arithmetic circuits, making the proof generation process computationally intensive and time-consuming.

We propose to ousource part of computations to a verification party, while still preserving user's privacy.



\section{Optimization}
Let $PK$ be a public key, $(r, s)$ - signature, $m$ - message, $G$ - base point (generator of EC group), $R$ - point, where $R.x == r$.
Then to verify an ECDSA signature, the following equation must hold:

\begin{equation}
    [m \cdot s^{-1}] G + [r \cdot s^{-1}] PK = R
\end{equation}

This equation consist of two scalar multiplications:
\begin{enumerate}
    \item $[m \cdot s^{-1}] G$ - scalar multiplication of base point;
    \item $[r \cdot s^{-1}] PK$ - scalar multiplication of public key.
\end{enumerate}

By using precompute tables and windowed double-and-add algorithm we can significantly optimize scalar multiplication of base point.
As the base point is known, the precomputation table can be generated during circuit compilation time, significantly reducing the number 
of constraints in the circuit.

The problem rises with scalar multiplication of public key. As it is not known beforehand, table should be calcutated in-circuit, which 
adds additional computational complexity.



Let $\mathcal{B}$ be random value selected uniformly from range $[0, 2^c)$, where $c$ is security parameter; 
$\mathbb{B} = [\mathcal{B}] PK$

By multiplying both sides of the equation by $\mathcal{B}$, we obtain:

\begin{equation}
  [m \cdot s^{-1} \cdot \mathcal{B}] G + [r \cdot s^{-1} \cdot \mathcal{B}] PK = [\mathcal{B}]R
\end{equation}

\begin{equation}
  [m \cdot s^{-1} \cdot \mathcal{B}] G + [r \cdot s^{-1}] \mathbb{B} = [\mathcal{B}]R
\end{equation}

There are 



\setcounter{tocdepth}{2}



\end{document}
