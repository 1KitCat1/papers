% Created 2024-09-23 Mon 13:35
% Intended LaTeX compiler: pdflatex
%%% TeX-command-extra-options: "-shell-escape"

\documentclass{iacrtrans}
\usepackage[utf8]{inputenc}
\usepackage[T1]{fontenc}

% -- Default Packages --
\usepackage{graphicx}
\usepackage{longtable}
\usepackage{wrapfig}
\usepackage{rotating}
\usepackage{tcolorbox}
\tcbuselibrary{skins}
\usepackage{booktabs}
% \usepackage{multirow}
\usepackage[normalem]{ulem}
\usepackage{amsmath}
\usepackage{amssymb}
\usepackage{capt-of}
\usepackage{hyperref}
\usepackage[nameinlink]{cleveref}
\crefname{algocf}{alg.}{algs.}
\Crefname{algocf}{Algorithm}{Algorithms}
\usepackage{tikz}
\usetikzlibrary{shapes.geometric, arrows.meta, positioning, trees}
\usepackage{forest}
\forestset{path highlight/.style={blue,edge={blue, line width=2pt},
for ancestors={edge={blue,line width=2pt},
  if level=0{}{blue}}}}
\usepackage{algpseudocode}
\usepackage{algorithm}

\newtcolorbox{scriptPubKeyBox}{
  enhanced,
  boxsep=1pt,
  arc=0.75ex,
  colback=gray!10,
  colframe=gray!40,
  boxrule=1pt,
  leftrule=80pt,
  top=-3.5mm,
  overlay unbroken and first ={%
    \node[minimum width=1cm,
      anchor=south,
      font=\sffamily\bfseries,
      xshift=40pt,
      yshift=-6.5pt,
    black]
    at (frame.west) {scriptPubKey:};
  }
}

\newtcolorbox{scriptSigBox}{
  enhanced,
  boxsep=1pt,
  arc=0.75ex,
  colback=gray!10,
  colframe=gray!40,
  boxrule=1pt,
  leftrule=80pt,
  top=-3.5mm,
  overlay unbroken and first ={%
    \node[minimum width=1cm,
      anchor=south,
      font=\sffamily\bfseries,
      xshift=40pt,
      yshift=-6.5pt,
    black]
    at (frame.west) {scriptSig:};
  }
}

\newcommand{\mycomment}[1]{}
\newcommand{\elem}[1]{\, \langle #1 \rangle \,}
\newcommand{\opcode}[1]{\, \texttt{#1} \,}
\newcommand{\script}[1]{ $\big\{ #1 \big\}$ }

% -- Algorithms --
\usepackage[
    titlenumbered,
    linesnumbered,
    ruled
]{algorithm2e}
\SetKwInOut{Input}{Input}
\SetKwInOut{Output}{Output}
\SetKwInOut{Return}{Return}

\SetKwComment{Comment}{/* }{ */}

\DeclareMathOperator*{\argmax}{arg\,max}
\DeclareMathOperator*{\argmin}{arg\,min}

\author{Scooby-Doo}
\institute{Distributed Lab}
\title{Optimized ZK verification of ECDSA signature with partially outsourced computations}

\usepackage{biblatex}
\addbibresource{refs.bib}

\begin{document}

\maketitle


\section{Introduction}
The Elliptic Curve Digital Signature Algorithm (ECDSA) is one of the most widely adopted signature schemes, particularly in blockchain systems. 
However, when it comes to zero-knowledge proof systems, verifying ECDSA signatures presents significant computational challenges.

Current implementations of ECDSA signature verification in zero-knowledge circuits face several limitations. 
The primary challenge lies in the computational complexity of the verification process, which requires numerous elliptic curve operations and modular arithmetic calculations. 
These operations translate into a large number of constraints when expressed in arithmetic circuits, making the proof generation process computationally intensive and time-consuming.

We propose to ousource part of computations to a verification party, while still preserving user's privacy.


\section{Problem overview}
Let $PK$ be a public key, $(r, s)$ - signature, $m$ - message, $G$ - base point (generator of EC group), $R$ - point, where $R.x == r$.
Then to verify an ECDSA signature, the following equation must hold:

\begin{equation}
    [m \cdot s^{-1}] G + [r \cdot s^{-1}] PK = R
\end{equation}

This equation consist of two scalar multiplications:
\begin{enumerate}
    \item $[m \cdot s^{-1}] G$ - scalar multiplication of base point;
    \item $[r \cdot s^{-1}] PK$ - scalar multiplication of public key.
\end{enumerate}

By using precompute tables and windowed double-and-add algorithm we can significantly optimize scalar multiplication of \textit{base point}.
As the base point is known, the precomputation table can be generated during circuit compilation time, significantly reducing the number 
of constraints in the circuit.

The problem rises with scalar multiplication of \textit{public key}. As it is not known beforehand, table should be calcutated in-circuit, which 
adds additional computational complexity.


Let $\mathbb{A}$ be number of constraints for point addition, $\mathbb{D}$ - for point double; $F$ - field size,
 $W$ - window size for windowed double and add.
Then the complexity of
\textit{base point} scalar multiplication is 
\[(\dfrac{F}{W} - 1) \cdot \mathbb{A};\]

for \textit{public key} scalar multiplication is 
\[(F - W) \cdot \mathbb{D} + (\dfrac{F - W}{W}) \cdot \mathbb{A} + (2 ^ W - 1) \cdot \mathbb{A} + \mathbb{D}\],

where $(2 ^ W - 1) \cdot \mathbb{A} + \mathbb{D}$ is required to generate a precompute table.


The complexity of scalar multiplication of public key is significantly higher than base point multiplication.


\textit{
With window size $W = 4$ and field size $F = 256$, the number of constraints for base point multiplication is
$63 \cdot \mathbb{A}$, while for public key multiplication it is $252 \cdot \mathbb{D} + 63 \cdot \mathbb{A} + 15 \cdot \mathbb{A} + \mathbb{D}$.
Additionally, the window size for \textit{base point} can be increased, resulting in lowering the number of constraints, but increasing off-circuit 
precomputation load.
}

\begin{table}[h]
\centering
\begin{tabular}{lr}
\hline
\textbf{Operation} & \textbf{Constraints} \\
\hline
Base point multiplication & 122,383 \\
Public key multiplication & 1,428,821 \\
\hline
\end{tabular}
\caption{Number of constraints for scalar multiplication operations in secp256r1}
\label{tab:constraints}
\end{table}

This shows that public key multiplication requires significantly more constraints, especially considering that point doubling ($\mathbb{D}$)
is typically more expensive than point addition ($\mathbb{A}$).


\section{Proposed optimization}

The intuition is to outsource scalar multiplication to the contract. If computing $[r \cdot s^{-1}] PK$ in-circuit requires too many constraints
we can commit to the result of such computations publicly and the smart contract would verify it:


\begin{algorithm}[H]
  \caption{ECDSA signature verification with outsourced scalar multiplication}
  \begin{algorithmic}[1]
  \Require $m$ - message hash, $r,s$ - signature components, $PK$ - public key
  \State $u_1 \gets m \cdot s^{-1} \bmod n$
  \State $u_2 \gets r \cdot s^{-1} \bmod n$
  \State $U_1 \gets [u_1]G$  \Comment{Computed in-circuit}
  \State $U_2 \gets [u_2]PK$  \Comment{Computed off-circuit}
  \State $R \gets U_1 + U_2$
  \State $R.x == r$
  
  \end{algorithmic}
  \end{algorithm}
  
Note that in such a scheme, values $u_2$ and $PK$ must be made public. 
However, revealing the public key directly may not be suitable for all applications, particularly those requiring privacy or anonymity.
To address this privacy concern, we propose using blinding values for off-circuit computation to obscure the actual public key 
while maintaining the correctness of the verification.


Let $\mathcal{B}$ be random value selected uniformly from range $[0, 2^c)$, where $c$ is security parameter; 
$\mathbb{B} = [\mathcal{B}] PK$

By multiplying both sides of the equation by $\mathcal{B}$, we obtain:

\begin{equation}
  [m \cdot s^{-1} \cdot \mathcal{B}] G + [r \cdot s^{-1} \cdot \mathcal{B}] PK = [\mathcal{B}]R
\end{equation}

\begin{equation}
  [m \cdot s^{-1} \cdot \mathcal{B}] G + [r \cdot s^{-1}] \mathbb{B} = [\mathcal{B}]R
\end{equation}


\begin{algorithm}[H]
  \caption{ECDSA signature verification with blinded public key}
  \begin{algorithmic}[1]
  \Require $m$ - message hash, $r,s$ - signature components, $PK$ - public key, $\mathcal{B}$ - blinding factor
  \State $u_1 \gets m \cdot s^{-1} \bmod n$
  \State $u_2 \gets r \cdot s^{-1} \bmod n$
  \State $\mathbb{B} \gets [\mathcal{B}]PK$ * 
  \State $U_1' \gets [u_1 \cdot \mathcal{B}]G$
  \State $U_2' \gets [u_2]\mathbb{B}$  \Comment{Computed off-circuit}
  \State $R' \gets [\mathcal{B}]R$ * 
  \State $U_1' + U_2' == R'$
  \end{algorithmic}
\end{algorithm}

* Note that computations of type $\mathbb{B} \gets [\mathcal{B}]PK$ is also heavy to compute and can be splitted to a different circuit.

\begin{algorithm}[H]
  \caption{Proving scalar multiplication of a point}
  \begin{algorithmic}[1]
  \Require private: $\mathcal{B}, PK$, public: $\mathbb{B}, commit$
  \State $\mathbb{B} \gets [\mathcal{B}]PK$
  \State $commit = Poseidon(\mathcal{B}, PK)$
  \end{algorithmic}
\end{algorithm}

The complete verification process can be represented as follows:

\begin{algorithm}[H]
  \caption{Proving scalar multiplication of a point}
  \begin{algorithmic}[1]
  \Require private: $m$ - message hash, $r,s$ - signature, $PK$ - public key, $\mathcal{B}$ - blinding factor

  public: $\mathbb{B}$, $commit$, $U_2', u_2,
  \pi$ - proof according to $Algorithm 3$
  \State $commit' = Poseidon(\mathcal{B}, PK)$
  \State $commit' == commit$
  \State $Verify(\pi, \mathbb{B}, \mathcal{B}, PK, commit)$ \Comment{Computed off-circuit}
  \State $u_1 \gets m \cdot s^{-1} \bmod n$
  \State $u_2 \gets r \cdot s^{-1} \bmod n$
  \State $U_2' \gets [u_2]\mathbb{B}$  \Comment{Computed off-circuit}
  \State $U_1' \gets [u_1 \cdot \mathcal{B}]G$
  \State $R' \gets [\mathcal{B}]R$
  \State $U_1' + U_2' == R'$
  \end{algorithmic}
\end{algorithm}


\section{Benchmarks}
/*TODO/*


\end{document}
